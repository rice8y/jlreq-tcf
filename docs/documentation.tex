\documentclass[a4paper]{article}
\usepackage{graphicx}
\usepackage{lmodern}
\usepackage{listings}
\usepackage{xcolor}
\usepackage{hologo}
\usepackage{url}

\newcommand{\pkgname}{\textsf{jlreq-tcf}}
\title{\pkgname: Two-column footnotes for \textsf{jlreq}}
\author{Eito Yoneyama}
\date{2026/01/08}

\lstset{
    basicstyle=\ttfamily\small,
    backgroundcolor=\color{gray!10},
    frame=single,
    breaklines=true,
    columns=fullflexible,
    language=TeX,
    keywordstyle=\color{blue},
    commentstyle=\color{olive},
}

\begin{document}

\maketitle

\begin{abstract}
    The \pkgname\ package provides a mechanism to typeset footnotes in two columns (left and right) at the bottom of the page. It is specifically designed to visually emulate the footnote style of the Japanese document class \textsf{jlreq}.
\end{abstract}

\tableofcontents

\section{Introduction}
Standard \LaTeX\ footnotes generally stack vertically in a single column. In Japanese typesetting contexts, it is sometimes desirable to split footnotes into two columns to efficiently use vertical space. This package provides \texttt{\textbackslash footnoteA} (left column) and \texttt{\textbackslash footnoteB} (right column) to achieve this layout.

\section{Installation}
\subsection{Using l3build}
This package is set up for \texttt{l3build}. To install it into your local texmf tree:
\begin{lstlisting}[language=bash]
l3build install
\end{lstlisting}

\subsection{Manual Installation}
Alternatively, ensure \texttt{jlreq-tcf.sty} is placed in a directory visible to \LaTeX\ (e.g., your local \texttt{texmf} tree or the same directory as your project \texttt{.tex} file)..

\section{Usage}

To use this package, load it in your preamble. Note that this package is intended to be used with the \textsf{jlreq} class.

\begin{lstlisting}
\documentclass{jlreq}
\usepackage{jlreq-tcf}
\end{lstlisting}

\section{Command Specifications}

This package uses its own internal counter (shared between A and B) to manage footnote numbers.

\subsection{Basic Footnote Commands}
These commands allow you to insert footnotes directly, similar to the standard \texttt{\textbackslash footnote} command.

\begin{description}
    \item[\texttt{\textbackslash footnoteA\{<text>\}}]
    Typesets a footnote in the \textbf{left-hand} column.
    \begin{itemize}
        \item Automatically increments the footnote counter.
        \item Places a footnote mark at the current location.
        \item Adds \texttt{<text>} to the left column queue at the bottom of the page.
    \end{itemize}
    
    \item[\texttt{\textbackslash footnoteB\{<text>\}}]
    Typesets a footnote in the \textbf{right-hand} column.
    \begin{itemize}
        \item Same behavior as \texttt{\textbackslash footnoteA}, but targets the right column.
    \end{itemize}
\end{description}

\subsection{Separated Mark and Text}
You can separate the mark generation from the text definition. This is useful when placing footnotes in titles, tables, or other restricted environments.

\begin{description}
    \item[\texttt{\textbackslash footnotemarkA[<num>]}]
    Prints the footnote mark for the left column.
    \begin{itemize}
        \item \textbf{With \texttt{[<num>]}:} Uses the integer \texttt{<num>} as the mark number. Does \textbf{not} increment the internal counter.
        \item \textbf{Without \texttt{[<num>]}:} Increments the internal counter and uses that value.
    \end{itemize}
    
    \item[\texttt{\textbackslash footnotemarkB[<num>]}]
    Prints the footnote mark for the right column. Arguments behave the same as \texttt{\textbackslash footnotemarkA}.

    \item[\texttt{\textbackslash footnotetextA[<num>]\{<text>\}}]
    Defines the footnote text for the left column without printing a mark in the main text.
    \begin{itemize}
        \item \textbf{With \texttt{[<num>]}:} Uses \texttt{<num>} as the label in the footnote area.
        \item \textbf{Without \texttt{[<num>]}:} Uses the current value of the internal counter.
    \end{itemize}

    \item[\texttt{\textbackslash footnotetextB[<num>]\{<text>\}}]
    Defines the footnote text for the right column. Arguments behave the same as \texttt{\textbackslash footnotetextA}.
\end{description}

\section{Visual Example}

Since this documentation is typeset in the \texttt{article} class, the example below is included from a separate file compiled with \textsf{jlreq} to demonstrate the actual layout.

\subsection{Source Code (example.tex)}
\begin{lstlisting}[language=TeX]
\documentclass{jlreq}
\usepackage{jlreq-tcf}
\usepackage{bxjalipsum}

\begin{document}
% \footnoteA
\jalipsum[1]{kusamakura}~\footnoteA{左の脚注1}\par
\jalipsum[2]{kusamakura}~\footnoteA{左の脚注2}\par
\jalipsum[3]{kusamakura}~\footnoteA{左の脚注3}

% \footnoteB
\jalipsum[4]{kusamakura}~\footnoteB{右の脚注1}\par
\jalipsum[5]{kusamakura}~\footnoteB{右の脚注2}\par
\jalipsum[6]{kusamakura}~\footnoteB{右の脚注3}

% \footnotemarkA
\jalipsum[7]{kusamakura}~\footnotemarkA

% \footnotemarkB
\jalipsum[8]{kusamakura}~\footnotemarkB

% \footnotetextA
\footnotetextA[7]{左の脚注4}

% \footnotetextB
\footnotetextB[8]{右の脚注4}
\end{document}
\end{lstlisting}

\subsection{Output}
The following page shows the result of the code above.

\begin{center}
  \fbox{\includegraphics[width=1.05\linewidth]{example.pdf}}
\end{center}

\section{Limitations and Known Issues}

Please carefully read the following limitations before using this package.

\subsection{Layout Emulation}
This package strives to reproduce the visual layout (rule width, spacing, indentation) of the \textsf{jlreq} class. However, please note that this is an \textbf{emulation}. The package manually constructs the footnote area using \texttt{minipage} environments rather than utilizing the internal hooks of the \textsf{jlreq} class. Consequently, slight differences in spacing or behavior may occur compared to native \textsf{jlreq} footnotes.

\subsection{Coexistence with Standard Footnotes}
\textbf{Do not use} the standard \texttt{\textbackslash footnote} command on the same page as \texttt{\textbackslash footnoteA} or \texttt{\textbackslash footnoteB}.

Since this package constructs a custom container for its footnotes, using the standard command concurrently will cause two separate footnote blocks to be printed at the bottom of the page: one generated by the standard \LaTeX\ output routine, and another generated by this package. We do not provide a workaround for this, as the package assumes you will use the two-column format exclusively for pages where it is active.

\subsection{Long Footnotes}
Typesetting extremely long footnotes is risky. Since the columns are wrapped in \texttt{minipage} environments to control the layout, text that exceeds the page height cannot break naturally across pages. This may result in the layout breaking or content disappearing off the bottom of the page.

\section{License}
This package is distributed under the BSD 2-Clause License. See the included LICENSE file.

\end{document}